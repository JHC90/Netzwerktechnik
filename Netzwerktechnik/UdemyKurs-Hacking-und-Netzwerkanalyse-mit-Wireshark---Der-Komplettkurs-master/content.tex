\documentclass[]{article}
\usepackage{hyperref} % für die Hyperlinks
%opening
\title{Content Wire-Shark-Course}
\author{Jochen Hollich}

\begin{document}

\maketitle

\begin{abstract}

\end{abstract}
\newpage

\section{Content}
%\begin{tabular}[h]{|c|c|c|}
\begin{tabular}{ p{2cm}|p{2cm}|p{2,1cm}|p{6cm}| }
	
	Chapter & Titel & Dokument & Beschreibung \\
	\hline
	Chapter 2 &
	Einfacher Mitschnittfilter&
	 \href{https://www.udemy.com/course/hacking-und-netzwerkanalyse-mit-wireshark-der-komplettkurs/learn/lecture/6851614#overview}{Video 13} 
	 \href{./Kapitel02/video13.pdf}{Document 13}  
	 & 
\begin{itemize}
	\item Capture filtered Traffic mit GUI
	\item Capture ICMP Traffic
	\item Drill down Capture ICMP Traffic to dst \& Src
	\item Capture DNS Traffic generiert über dig
	\item Drill Down Capture DNS Traffic auf die Hosts = Filterkombination
	\item zDrill Down Capture DNS Traffic auf die Hosts = Filterkombination mittels logischer Verknüpfungen AND | OR
	\item klären wie mit logischer Verknüpfung dann AND und OR usw verwendet werden
\end{itemize}

	
	 
	 \\
	 
		\hline
	Chapter 2 &
	IP-Adress Auflösung via ARP&
	\href{https://www.udemy.com/course/hacking-und-netzwerkanalyse-mit-wireshark-der-komplettkurs/learn/lecture/6851616#announcements}{Video 14} 
	\href{./Kapitel02/video14.pdf}{Document 14}  
	& 
	\begin{itemize}
		\item Capture filtered Traffic mit GUI
		\item Capture ICMP Traffic
		\item Drill down Capture ICMP Traffic to dst \& Src
		\item Capture DNS Traffic generiert über dig
		\item Drill Down Capture DNS Traffic auf die Hosts = Filterkombination
		\item zDrill Down Capture DNS Traffic auf die Hosts = Filterkombination mittels logischer Verknüpfungen AND | OR
	\end{itemize}
	
	
	
	
	
	
	\label{tab:heisetabelle}
\end{tabular}




\end{document}
