\documentclass[]{article}

% laden der Packages
\usepackage{hyperref} % für die Hyperlinks
\usepackage{graphicx} % für die Titlegrafik
\usepackage{titling} % für die Titelgrafik
\usepackage{color} % für die Titelgrafik

\usepackage[ansinew]{inputenc} % coiduerng

\usepackage{xcolor}% für Farbliche Markeriung
\definecolor{hellgrau}{gray}{0.85} % für Farbliche Markeriung
\usepackage{soul} % für Farbliche Markeriung
\newcommand{\opt}[1]{\texttt{#1}}% für Farbliche Markeriung
\newcommand{\opti}[1]{\mbox{}{\color{blue}\texttt{#1}}}% für Farbliche Markeriung

%opening
\title{Video 14}
\author{Jochen Hollich}
\renewcommand\maketitlehooka{%
	\begin{center}
		\fbox{\includegraphics[width=1\textwidth]{./imgs/Titelbild.jpg}}
	\end{center}%
}


\begin{document}
	
	\maketitle % der titel wird zwar darüberliegend definiert, wenn die Zeile aber nicht inkludiert wird sie im PDF nicht gerendert
	
	\begin{abstract}
		Hier geht es um die Mitschrift aus \href{https://www.udemy.com/course/hacking-und-netzwerkanalyse-mit-wireshark-der-komplettkurs/learn/lecture/6851616#announcements}{Video 14} mit dem Titel "IP-Adressauflösung mit ARP" .
	\end{abstract}
	
	\newpage
	
	\section{Mitschrift}
	Hier die Punkte um welche wir uns gekümmert haben:
	\begin{description}
		\item Filter auf DNS-Traffic auf einen bestimmten port  \\ 
		{\sethlcolor{cyan}\hl{udp dst port 53}} \\
		{\sethlcolor{yellow}\hl{dig @8.8.8.8 www.cbt-24.de A +tcp}} \\ 
		hier nur den Traffic von udp \& dst 
	\end{description}
	
\end{document}
